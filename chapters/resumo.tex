\section*{Resumo}
\addcontentsline{toc}{section}{Resumo}
Os formulários digitais são um método popular para recolha de dados.
O suporte para validações em tempo real que filtram dados inválidos torna-os particularmente desejáveis.
Validações baseadas em expressões regulares são usadas frequentemente em formulários digitais para evitar que os utilizadores introduzam dados no formato errado.
No entanto, a escrita destas validações pode representar um desafio para alguns utilizadores.

Neste documento, apresentamos o \Forest, um sintetizador de expressões regulares para validação de formulários digitais.
O \Forest produz uma expressão regular que corresponde ao padrão desejado para os valores de \textit{input},
um conjunto de grupos de captura que permitem extrair deles mais informação,
e
um conjunto de condições sobre grupos de captura que garantem a validade de valores inteiros no \textit{input}.
O nosso método de síntese é baseado em procura enumerativa e usa um \textit{solver} de Satisfazibilidade Módulo Teorias (SMT) para explorar e podar o espaço de procura.
Propomos uma nova representação para a síntese de expressões regulares, multi-árvore, que induz padrões nos exemplos e os usa para dividir o problema através de uma abordagem dividir para conquistar.
Apresentamos ainda uma nova codificação SMT para sintetizar as condições sobre capturas para uma dada expressão regular.
Para aumentar a confiança na expressão regular sintetizada, implementamos um modelo interação com o utilizador com base em \textit{inputs} distintivos.

Avaliámos o \Forest{} em instâncias de validação de formulários do mundo real com base em expressões regulares. Os resultados experimentais mostram que o \Forest{} retorna com sucesso a expressão regular desejada em 72\% das instâncias e supera o \Regel, um sintetizador de expressões regulares estado-da-arte.

\vfill
\noindent
\textbf{\Large Palavras-chave:} Síntese de programas, Programação-por-Exemplo, Satisfazibilidade Módulo Teorias, Expressões Regulares, Validação de Formulários.

\cleardoublepage