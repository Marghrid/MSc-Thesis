
\subsection{Regular Definitions}

For convenience, we may wish to give names to certain regular expressions and use those names in subsequent expressions, as if the names were themselves symbols. If \(\Sigma\) is an alphabet of basic symbols, then a regular definition is a sequence of definitions of the form:

\begin{equation}
\begin{gathered} % centered multiline equation.
d_1 \to r_1\\
d_2 \to r_2\\
...\\
d_n \to r_n
\end{gathered}
\end{equation}

\noindent
where each \(d_i\) is a new symbol, not in \(\Sigma\) and not the same as any of the other \(d\), and each \(r_i\) is a regular expression over the alphabet \(\Sigma \cup \{ d_1, d_2, ..., d_{i-1} \}\).

By restricting \(r_i\) to \(\Sigma\) and the previously defined \(d\)'s, we avoid recursive definitions, and we can construct a regular expression over \(\Sigma\) alone, for each \(r_i\).
We do so by first replacing uses of \(d_1\) in \(r_2\) (which cannot use any of the d's except
for \(d_1\)), then replacing uses of \(d_1\) and \(d_2\) in \(r_3\) by \(r_1\) and (the substituted) \(r_2\), and so on. Finally, in \(r_n\) we replace each \(d_i\), for \(i = 1, 2, ..., n-1\), by the substituted version of \(r_i\), each of which has only symbols of \(\Sigma\).

\begin{example}

We may wish to
\begin{equation*}
\begin{aligned}
\regex{number} &\to \regex{0|1|2|3|4|5|6|7|8|9} \\
%
\regex{lower\_case} &\to \regex{a|b|c|d|e|f|g|h|i|j|k|l|m|n|o|p|q|r|s|t|u|v|w|x|y|z} \\
%
\regex{upper\_case} &\to \regex{A|B|C|D|E|F|G|H|I|J|K|L|M|N|O|P|Q|R|S|T|U|V|W|X|Y|Z}\\
%
\regex{letter} &\to \regex{upper\_case | lower\_case}
\end{aligned}
\end{equation*}
\end{example}




\subsection{Regex vs \acp{CFG}} \label{sec:regex-vs-CFG}

Grammars are a more powerful notation than regular expressions. Every construct that can be described by a regular expression can be described by a grammar, but not vice-versa. Alternatively, every regular language is a context-free language, but not vice-versa.

\begin{example}
Consider the regular expression \((a|b)^*abb\) and the \ac{CFG}:

\begin{equation}
\begin{aligned} % centered multiline equation.
A_0 &\to aA_0 | bA_0 | aA_1\\
A_1 &\to bA_2\\
A_2 &\to bA_3\\
A_3 &\to \epsilon
\end{aligned}
\end{equation}

The regular expression and the grammar describe the same language: the set of strings in \(\{a, b\}\) that end in \(abb\).

\end{example}

\begin{example}
The language \(\mathcal{L} = \{a^n b^n : n \ge 1\}\) is the language of strings in \(\{a, b\}\) that start with some \(a\)s followed by the same number of \(b\)s.  \(\mathcal{L}\) can be defined using a \ac{CFG} but not using a regular expression:  \(\mathcal{L}\) is a context-free language but not a regular language.
\end{example}






\section{Form Input Validation Frameworks} \label{sec:form-validation}

In order to get some insight into the kind of operations and structures we would need to synthesise within the domain of form validation, we looked into three different tools which provide several types of validations over the input~fields:

\paragraph{Google forms\protect\footnotemark} is a survey administration app that allows a user to build personalised surveys or quizzes. The validations are added to the different input fields using a graphical interface. 
\footnotetext{\url{https://www.google.com/forms/about/}. Accessed on \formatdate{1}{1}{2020}.}

\paragraph{HTML Input Attributes\protect\footnotemark}  are specified within an \texttt{HTML} form, inside the \texttt{<input>} element. Until all the values comply with the requirements introduced by these specifications, the form cannot be submitted.
\footnotetext{\url{https://www.w3schools.com/html/html_form_attributes.asp}. Accessed on \formatdate{1}{1}{2020}.}

\paragraph{ActiveRecord Validations\protect\footnotemark}  is a \textit{Ruby on Rails} application that offers many predefined validation helpers, which are applied to the form values before they are submitted into a database.

\footnotetext{\url{https://edgeguides.rubyonrails.org/active_record_validations.html}. Accessed on \formatdate{1}{1}{2020}.}

\medskip
\noindent
Although these frameworks' interfaces and intended usage differ, the style of validations offered is similar. The following validations are common to all these frameworks:

\begin{itemize}
  \item \textbf{Required}: Specifies that an input field cannot be blank;
  \item \textbf{Type}: Specifies the type of an input value. Possible values include integer, text, email, url, date;
  \item \textbf{Min/Max}: Specifies the minimum/maximum value of a numeric input value;
  \item \textbf{Min-/Maxlength}: Specifies the minimum/maximum length of a string input value;
  \item \textbf{Pattern match}: Specifies a regular expression against which an input value is checked.
\end{itemize}

\noindent
The validations are applied individually on the input fields. Each field can have more than one validation, and one validation can be in respect to more than one input field. The form is valid when all validations on all input fields are verified.

\SaveVerb{bookprice}=[1-9]+[0-9]*(\.[0-9][0-9]?)?=

\begin{example} \label{ex:book-validation}
\parindent0pt
Suppose we build a form to ask for information about a book with three input fields: \textit{book name}, \textit{number of pages} and \textit{book price}. 

If we want to ensure that the name is between 2 and 50 characters long, we may do so by adding the following validations over input field \(\mathit{book\,name}\):
\begin{equation*}
\begin{split}
    &minlength(\mathit{book\,name}, 2),\;\text{and}\\
    &maxlength(\mathit{book\,name}, 50).
\end{split}
\end{equation*}

Then, to ensure the number of pages is a positive integer:
\begin{equation*}
\begin{gathered}
    type(\mathit{number\,of\,pages}, integer),\;\text{and}\\
    min(\mathit{number\,of\,pages}, 1).\\
\end{gathered}
\end{equation*}
Finally, we can use a regular expression to check that the price is a number with at most 2 decimal~places:
\[pattern\_match(\mathit{book\,price}, \UseVerb{bookprice})\]

If these 5 validations are verified by the respective input values, the form is valid.
\end{example}


\begin{figure}
    \centering
    \vspace{5ex}
    \begin{subfigure}[b]{0.47\textwidth}
        \begin{tikzpicture}
            \begin{axis}[xmode=log, ymode=log, grid=major,
                        xlabel = {\Forest},
                        ylabel = {\Regel},
                        xmax = 5000 ,ymax = 5000,
                        width=.5\linewidth, height=.5\linewidth,
                        clip marker paths=true]
            \addplot[thick, mark size=3pt, only marks, mark=x, fill opacity=0.6, draw opacity=0.9, color=scatter-color] table[x=forest regex time, y=regel time, col sep=comma] {data/regel_comparison.csv};
            \addplot[domain=0.01:3600,smooth,thin]{x};
            \addplot [dotted, thick, timeout-color, opacity=0.8] coordinates {(0.01,3600) (3600,3600)}
                node at (3,2500) {\scriptsize 3600 second timeout};
            \addplot [dotted, thick, timeout-color, opacity=0.8] coordinates {(3600,0.01) (3600,3600)}
                node [rotate=-90] at (2500,3) {\scriptsize 3600 second timeout};
            \end{axis}
        \end{tikzpicture}
        \caption{\Regel vs. \Forest on \Forest's instances.}
        \label{fig:regel_comparison}
    \end{subfigure}
    \hfill
    \begin{subfigure}[b]{0.47\textwidth}
        \begin{tikzpicture}
            \begin{axis}[xmode=log, ymode=log, grid=major,
                        xlabel = {\Forest},
                        ylabel = {\Regel},
                        xmax = 5000 ,ymax = 5000,
                        width=.5\linewidth, height=.5\linewidth,
                        clip marker paths=true]
            \addplot[thick, mark size=3pt, only marks, mark=x, fill opacity=0.6, draw opacity=0.9, color=scatter-color] table[x=forest regex time, y=regel time, col sep=comma] {data/regel_comparison.csv};
            \addplot[domain=0.01:3600,smooth,thin]{x};
            \addplot [dotted, thick, timeout-color, opacity=0.8] coordinates {(0.01,3600) (3600,3600)}
                node at (3,2500) {\scriptsize 3600 second timeout};
            \addplot [dotted, thick, timeout-color, opacity=0.8] coordinates {(3600,0.01) (3600,3600)}
                node [rotate=-90] at (2500,3) {\scriptsize 3600 second timeout};
            \end{axis}
        \end{tikzpicture}
        \caption{\Regel vs. \Forest on \Forest's instances.}
        \label{fig:regel_comparison}
    \end{subfigure}
    %\begin{subfigure}[b]{0.47\textwidth}
    %     \begin{tikzpicture}
    %         \begin{axis}[xmode=log, ymode=log, grid=major,
    %                     xlabel = {\Forest},
    %                     ylabel = {\Regel + 60},
    %                     xmax = 5000 ,ymax = 5000,
    %                     width=.99\linewidth, height=.99\linewidth,
    %                     clip marker paths=true]
    %         \addplot[only marks, mark=x, fill opacity=0.6, draw opacity=0.9, color=scatter-color] table[x=forest regex time, y=regel plus 60, col sep=comma] {data/regel_comparison.csv};
    %         \addplot[domain=0.1:3600,smooth,thin]{x};
    %         \addplot [dotted, thick, timeout-color, opacity=0.8] coordinates {(0.1,3600) (3600,3600)}
    %             node at (3,2500) {\tiny{3600 second timeout}};
    %         \addplot [dotted, thick, timeout-color, opacity=0.8] coordinates {(3600,0.1) (3600,3600)}
    %             node [rotate=-90] at (2500,3) {\tiny{3600 second timeout}};
    %         \end{axis}
    %     \end{tikzpicture}
    %     \caption{\Regel + sketch time vs. \Forest}
    %     \label{fig:regel_comparison}
    % \end{subfigure}
    % \par\vspace{3em}
    % \begin{subfigure}[b]{0.45\textwidth}
    %     \begin{tikzpicture}
    %         \begin{axis}[xmode=log, ymode=log, grid=major,
    %                     xlabel = {\Forest},
    %                     ylabel = {\Regel w/o sketches},
    %                     xmax = 5000 ,ymax = 5000,
    %                     width=.99\linewidth, height=.99\linewidth,
    %                     clip marker paths=true]
    %         \addplot[only marks, mark=x, fill opacity=0.6, draw opacity=0.9, color=scatter-color] table[x=forest regex time, y=regel no sketch, col sep=comma] {data/regel_comparison.csv};
    %         \addplot[domain=0.1:3600,smooth,thin]{x};
    %         \addplot [dotted, thick, timeout-color, opacity=0.8] coordinates {(0.1,3600) (3600,3600)}
    %             node at (3,2500) {\tiny{3600 second timeout}};
    %         \addplot [dotted, thick, timeout-color, opacity=0.8] coordinates {(3600,0.1) (3600,3600)}
    %             node [rotate=-90] at (2500,3) {\tiny{3600 second timeout}};
    %         \end{axis}
    %     \end{tikzpicture}
    %     \caption{\Regel without sketches vs. \Forest}
    %     \label{fig:regel_comparison}
    %\end{subfigure}
\end{figure}

\begin{table}[t]
\centering
\setlength{\tabcolsep}{2ex}
\caption{Comparison of time performance of \Regel{} and \Forest{} on \Regel{}'s Stack Overflow benchmarks.}
\begin{tabular}{@{}lcccc@{}}
\toprule
\textbf{Timeout (s)} & \textbf{10} & \textbf{60} & \textbf{3600} \\ \midrule
\Regel{} PBE               & 63 & 73 & 92 \\
\Forest{}'s first solution & 34 & 47 & 60 \\
\Forest{}'s last solution  & 17 & 28 & 50 \\ \bottomrule
\end{tabular}
\label{table:number-solved-resnax}
\vspace*{-2mm}
\end{table}

\todo{Comparison with \Regel:}
\color{blue}
\autoref{table:number-solved-resnax} shows the number of instances solved by each approach. \Forest{} DSL only supports 95 out of 122 instances. After 1 hour, \Forest{} could only return a solution to 50 of \Regel{}'s benchmarks (41\%). In 10 other benchmarks, after 1 hour, \Forest{} had already found a solution consistent with the examples, but it was still disambiguating. Of the synthesised regexes, only 13 are correct considering the natural language description provided along with the examples (11\%). In the same amount of time, \Regel{} returned 92 regexes (75\%), of which 16 (13\%) satisfy the natural language description. Note that \Regel{} can also take as input a natural language description of the problem and their evaluation shows that for this set of benchmarks, the accuracy of \Regel{} is increased substantially when using natural language~\cite{Regel20}. However, the same behaviour could also be expected if \Forest{} would be extended with this kind of information.

This comparison shows the PBE engine of \Forest{} is much better than \Regel{} for form validation benchmarks where it exploits the structure of the data. For the synthesis of regexes where there is no structure, we observe that the performance of both PBE engines is comparable.
\color{black}