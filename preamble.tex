\usepackage[utf8]{inputenc}
\usepackage[british]{babel}
\usepackage[margin=2.5cm]{geometry}
\usepackage{float}
\usepackage[nodayofweek]{datetime} % format dates
\newdateformat{monthyeardate}{\monthname[\THEMONTH], \THEYEAR}
\usepackage{setspace}
\renewcommand{\baselinestretch}{1.5}

\usepackage{graphicx}
\graphicspath{{Pictures/}}
\usepackage[dvipsnames]{xcolor}

\usepackage{tikz}
\usetikzlibrary{shapes.geometric, arrows, positioning, calc, backgrounds, trees, shadings}
% Numbers in log scales are in decimal form

\definecolor{synthesizer-color}{HTML}{adcbe3}
\definecolor{enumerator-color}{HTML}{4b86b4}
\tikzset{%
    square/.style={
    rectangle,
    rounded corners,
    minimum width=2.5cm,
    minimum height=1cm,
    text centered,
    draw=darkgray,
    fill=gray!20
  },
  nosquare/.style={
    rectangle,
    rounded corners,
    % fill=gray!20,
    text centered
  },
  arrow/.style={
    thick,
    rounded corners=10,
    ->,>=stealth
  },
  enclosing_square/.style={
    draw=gray!50,
    very thick,
    dotted,
    fill=gray!10
  },
  synth_node/.style={
    synthesizer-color!30!gray
  }
}
\usepackage{pgfplots}
\pgfplotsset{log ticks with fixed point, compat=1.16}
\usepgfplotslibrary{colorbrewer}

\usepackage{amsmath}  % AMS mathematical facilities for LaTeX.
\usepackage{amsthm}   % Typesetting theorems (AMS style).
\usepackage{amsfonts}
\usepackage{amssymb}
\usepackage{url}
\usepackage[inline]{enumitem}
\usepackage{comment}
\usepackage{acronym}
\usepackage{fancyvrb}   % save verbatims
\usepackage{xspace}
\usepackage[style=base]{caption} % caption setup
\usepackage{subcaption}
\usepackage{nth}   % superscript ordinal numbers
\usepackage{algpseudocode}
\usepackage{algorithm}
\usepackage{listings}
\usepackage{afterpage} % clever \clearpage, puts the effect off until the page is full

\mathchardef\mhyphen="2D % hyphen in math mode


\addto\captionsenglish{\renewcommand{\acknowledgments}{Acknowledgments}}
%\addto\captionsenglish{\renewcommand{\contentsname}{Contents}}
%\addto\captionsenglish{\renewcommand{\listtablename}{List of Tables}}
%\addto\captionsenglish{\renewcommand{\listfigurename}{List of Figures}}
%\addto\captionsenglish{\renewcommand{\nomname}{Nomenclature}}
%\addto\captionsenglish{\renewcommand{\glossaryname}{Glossary}}
%\addto\captionsenglish{\renewcommand{\acronymname}{List of Acronyms}}
%\addto\captionsenglish{\renewcommand{\bibname}{References}}
%\addto\captionsenglish{\renewcommand{\appendixname}{Appendix}}

\addto\captionsbritish{\renewcommand{\acknowledgments}{Acknowledgments}}
%\addto\captionsbritish{\renewcommand{\contentsname}{Contents}}
%\addto\captionsbritish{\renewcommand{\listtablename}{List of Tables}}
%\addto\captionsbritish{\renewcommand{\listfigurename}{List of Figures}}
%\addto\captionsbritish{\renewcommand{\nomname}{Nomenclature}}
%\addto\captionsbritish{\renewcommand{\glossaryname}{Glossary}}
%\addto\captionsbritish{\renewcommand{\acronymname}{List of Acronyms}}
%\addto\captionsbritish{\renewcommand{\bibname}{References}}
%\addto\captionsbritish{\renewcommand{\appendixname}{Appendix}}

\newcommand{\coverThesis}{@undefined}
\newcommand{\coverSupervisors}{@undefined}
\newcommand{\coverExaminationCommittee}{@undefined}
\newcommand{\coverChairperson}{@undefined}
\newcommand{\coverSupervisor}{@undefined}
\newcommand{\coverMemberCommittee}{@undefined}

\addto\captionsenglish{\renewcommand{\coverThesis}{Thesis to obtain the Master of Science Degree in}}
\addto\captionsenglish{\renewcommand{\coverSupervisors}{Supervisor(s)}}
\addto\captionsenglish{\renewcommand{\coverExaminationCommittee}{Examination Committee}}
\addto\captionsenglish{\renewcommand{\coverChairperson}{Chairperson}}
\addto\captionsenglish{\renewcommand{\coverSupervisor}{Supervisor}}
\addto\captionsenglish{\renewcommand{\coverMemberCommittee}{Member of the Committee}}

\addto\captionsbritish{\renewcommand{\coverThesis}{Thesis to obtain the Master of Science Degree in}}
\addto\captionsbritish{\renewcommand{\coverSupervisors}{Supervisor(s)}}
\addto\captionsbritish{\renewcommand{\coverExaminationCommittee}{Examination Committee}}
\addto\captionsbritish{\renewcommand{\coverChairperson}{Chairperson}}
\addto\captionsbritish{\renewcommand{\coverSupervisor}{Supervisor}}
\addto\captionsbritish{\renewcommand{\coverMemberCommittee}{Member of the Committee}}

% Use Arial font as default
\renewcommand{\rmdefault}{phv}
\renewcommand{\sfdefault}{phv}

% Define cover page fonts

\def\FontLn{% 16 pt normal
  \usefont{T1}{phv}{m}{n}\fontsize{16pt}{16pt}\selectfont}
\def\FontLb{% 16 pt bold
  \usefont{T1}{phv}{b}{n}\fontsize{16pt}{16pt}\selectfont}
\def\FontMn{% 14 pt normal
  \usefont{T1}{phv}{m}{n}\fontsize{14pt}{14pt}\selectfont}
\def\FontMb{% 14 pt bold
  \usefont{T1}{phv}{b}{n}\fontsize{14pt}{14pt}\selectfont}
\def\FontSn{% 12 pt normal
  \usefont{T1}{phv}{m}{n}\fontsize{12pt}{12pt}\selectfont}

\usepackage{dcolumn}
\newcolumntype{d}{D{.}{.}{-1}} % column aligned by the point separator '.'
\newcolumntype{e}{D{E}{E}{-1}} % column aligned by the exponent 'E'


\usepackage{hyperref} % enhance documents that are to be
                              % output as HTML and PDF
\hypersetup{colorlinks,       % color text of links and anchors,
                              % eliminates borders around links
%            linkcolor=red,    % color for normal internal links
            linkcolor=black,  % color for normal internal links
            anchorcolor=black,% color for anchor text
%            citecolor=green,  % color for bibliographical citations
            citecolor=black,  % color for bibliographical citations
%            filecolor=magenta,% color for URLs which open local files
            filecolor=black,  % color for URLs which open local files
%            menucolor=red,    % color for Acrobat menu items
            menucolor=black,  % color for Acrobat menu items
%            pagecolor=red,    % color for links to other pages
%            urlcolor=cyan,    % color for linked URLs
             urlcolor=black,   % color for linked URLs
	          bookmarksopen=false,    % don't expand bookmarks
	          bookmarksnumbered=true, % number bookmarks
	          pdftitle={Thesis},
            pdfauthor={Andre C. Marta},
            pdfsubject={Thesis Title},
            pdfkeywords={Thesis Keywords},
            pdfstartview=FitV,
            pdfdisplaydoctitle=true}

\usepackage[figure,table]{hypcap}


%% natbib options can be provided when package is loaded \usepackage[options]{natbib}
%%
%% Following options are valid:
%%
%%   round  -  round parentheses are used (default)
%%   square -  square brackets are used   [option]
%%   curly  -  curly braces are used      {option}
%%   angle  -  angle brackets are used    <option>
%%   semicolon  -  multiple citations separated by semi-colon (default)
%%   colon  - same as semicolon, an earlier confusion
%%   comma  -  separated by comma
%%   authoryear - for author–year citations (default)
%%   numbers-  selects numerical citations
%%   super  -  numerical citations as superscripts, as in Nature
%%   sort   -  sorts multiple citations according to order in ref. list
%%   sort&compress   -  like sort, but also compresses numerical citations
%%   compress - compresses without sorting
%%
% ******************************* SELECT *******************************

\usepackage[numbers,sort&compress]{natbib}

\usepackage{multirow} % tabular cells spanning multiple rows

% \renewcommand{\arraystretch}{<ratio>} % space between rows
%
\usepackage{booktabs}
%\newcommand{\ra}[1]{\renewcommand{\arraystretch}{#1}}

% \usepackage{pdfpages}

\usepackage{arydshln} % dashed lines in tables

\usepackage{multicol}

\usepackage{placeins} % float barrier at the end of each section

\usepackage{makecell} % for multiline cells in tables

% ----------------------------------------------------------------------
% Define new commands to assure consistent treatment throughout document
% ----------------------------------------------------------------------

\theoremstyle{definition}
%definitions on subsections
\newtheorem{definition}{Definition}[section]

%examples on subsections
\newtheorem{example}{Example}[section]


\newcommand{\ud}{\mathrm{d}}                % total derivative
\newcommand{\degree}{\ensuremath{^\circ\,}} % degrees


% If I can't decide, at least I make it easy to change
\newcommand{\true} {\textit{True}\xspace}
\newcommand{\false}{\textit{False}\xspace}
\newcommand{\regex}[1]{\(\tt #1\)}  % To define regexs
\newcommand{\regexm}[1]{\texttt{#1}} % to define regexs inside math mode
\newcommand{\set}[1]{{\ensuremath{\mathcal{#1}}}}  % math calygraphy
\newcommand{\Forest}{\textsc{Forest}\xspace}
\newcommand{\AlphaRegex}{\textsc{AlphaRegex}\xspace}
\newcommand{\Regel}{\textsc{Regel}\xspace}
\newcommand{\UnchartIt}{\textsc{UnchartIt}\xspace}
\newcommand{\optmodel}{\textsc{Options}\xspace} % Ramos's interaction models
\newcommand{\ynmodel}{\textsc{Y/N}\xspace} % Ramos's interaction models


\newcommand{\todo}[1]{\textbf{\color{purple} [#1]}}
\newcommand{\revise}[1]{\textcolor{MidnightBlue}{#1}}
\newcommand{\mycomment}[1]{\textcolor{blue}{\textbf{[#1]}}}

\newcommand{\acknowledgments}{@undefined}

\SaveVerb{date}|[0-9][0-9]/[0-9][0-9]/[0-9][0-9][0-9][0-9]|
\SaveVerb{date2}|[0-9]{2}/[0-9]{2}/[0-9]{4}|
\SaveVerb{date_day_mo_caps}|([0-9]{2})/([0-9]{2})/[0-9]{4}|
\SaveVerb{date_year_cap}|[0-9]{2}/[0-9]{2}/([0-9]{4})|

\hyphenation{DD/MM/YYYY}

\newcommand{\IsInt}{\texttt{is\_int}}
\newcommand{\IsReal}{\texttt{is\_real}}
\newcommand{\IsString}{\texttt{is\_string}}
%
\newcommand{\Len}{\texttt{length}}
\newcommand{\Match}{\texttt{match}}
%
\newcommand{\Input}{\texttt{input}}
%
\newcommand{\Numlit}{\texttt{numlit}}
\newcommand{\Relit} {RegexLit}
\newcommand{\Rangelit} {RangeLit}
%
\newcommand{\Concat} {\texttt{concat}}
\newcommand{\Union} {\texttt{union}}
\newcommand{\Kleene} {\texttt{kleene}}
\newcommand{\Posit} {\texttt{posit}}
\newcommand{\Option} {\texttt{option}}
\newcommand{\Range} {\texttt{range}}
% this is a template scatter plot
\definecolor{scatter-color}{HTML}{0026d1}
\definecolor{timeout-color}{HTML}{ba0039}
\definecolor{error-color}{HTML}{a30b00}
\newcommand{\compareTimesPlot}[4]{
\begin{tikzpicture}
    \begin{axis}[xmode=log, ymode=log, grid=major,
                xlabel = {#2},
                ylabel = {#4},
                xmax = 5000 ,ymax = 5000,
                width=.99\linewidth, height=.99\linewidth,
                clip marker paths=true]
    \addplot[thick, mark size=3, only marks, mark=x, fill opacity=0.6, draw opacity=0.7, color=scatter-color] table[x=#1, y=#3, col sep=comma] {data/time_comparison.csv};
    %
    \addplot[domain=0.1:3600,smooth,thin]{x}; % y = x line
    %
    \addplot [dotted, thick, timeout-color, opacity=0.8] coordinates {(0.1,3600) (3600,3600)}
        node at (100,2500) {\scriptsize 3600 second timeout}; % y timeout
    %
    \addplot [dotted, thick, timeout-color, opacity=0.8] coordinates {(3600,0.1) (3600,3600)}
        node [rotate=-90] at (2500,1) {\scriptsize 3600 second timeout}; % x timeout
    \end{axis}
\end{tikzpicture}%
} % time x time plots


\setlist[enumerate]{itemsep=.5ex}

\setlist[itemize]{itemsep=.5ex}

\frenchspacing


% Redefine autoref automatic prefix.
% It is "Section" for sections, subsections and subsubsections.
\def\sectionautorefname{Section}
\def\sectionname{Section}
\def\subsectionautorefname{Section}
\def\subsectionname{Section}
\def\subsubsectionautorefname{Section}
\def\subsubsectionname{Section}
\def\exampleautorefname{Example}
\def\examplename{Example}