\chapter{Capturing Groups Synthesis}

\todo{Motivation and introduction to subsections}

In this chapter we describe the synthesis procedure that generates the second and third components of the regex validation:
\begin{itemize}
    \item \textbf{capturing groups} that reflect the captures provided with the valid examples;
    \item \textbf{capture conditions} which express integer conditions for the values in the example that are satisfied by all valid examples but not by any of the conditional invalid examples. 
\end{itemize}

    
\section{Capturing Groups Enumeration}\label{sec:cap_groups_enumeration}
The synthesis of both the capturing groups and the capture conditions is done using enumerative search. Thus, the first step in both synthesis procedures is to enumerate capturing groups over the previously computed regular expression.

To enumerate capturing groups, \Forest starts by identifying atomic sub-regexes in the regular expression.
Atomic regexes correspond to the smallest sub-regexes whose concatenation results in the original complete regex.
It does not make sense to place a capturing group inside atomic sub-regexes. For example, the regex \verb![0-9]{2}! is an atomic sub-regex: there are no smaller sub-regexes whose concatenation results in it. It does not make sense to place a capturing group on just a part of~it: \verb!([0-9]){2}!.
% In the tree representation of regexes, an atomic sub-regex is any subtree of regex whose root node is not a concatenation and whose parents are nothing but concatenations.
Once identified, the atomic sub-regexes are placed in an ordered list. The concatenation of all elements in this list results in the original regular expression.

\begin{example}
Recall the previously shown date regex: \UseVerb{date2}. The ordered list of atomic sub-regexes for this regex is [\verb![0-9]{2}!, \verb!/!, \verb![0-9]{2}!, \verb!/!, \verb![0-9]{4}!].
\end{example}

Enumerating capturing groups over the regular expression is done by enumerating non-empty disjoint sub-lists of this list. The elements inside each sub-list form a capturing group. \todo{Do I need to show the generator I use to enumerate disjoint sub-lists? It's here: \url{https://www.overleaf.com/1387546481sndgbkmsbjgj}}

\begin{example}
Suppose we want to get a single capturing group within the date regex. % \UseVerb{date2}.
To do so, we use the ordered list of atomic sub-regexes [\verb![0-9]{2}!, \verb!/!, \verb![0-9]{2}!, \verb!/!, \verb![0-9]{4}!].
The following are some examples of sub-lists of the atomic sub-regexes list and their resulting capturing groups:

%\begin{table}[h!]
%\centering
\begin{center}
\setlength{\extrarowheight}{10pt}
\begin{tabular}{rcl}
[[\verb![0-9]{2}!], \verb!/!, \verb![0-9]{2}!, \verb!/!, \verb![0-9]{4}!] & $\rightarrow$ & \verb!([0-9]{2})/[0-9]{2}/[0-9]{4}! \\

[[\verb![0-9]{2}!, \verb!/!], \verb![0-9]{2}!, \verb!/!, \verb![0-9]{4}!] & $\rightarrow$ & \verb!([0-9]{2}/)[0-9]{2}/[0-9]{4}! \\

[\verb![0-9]{2}!, \verb!/!, [\verb![0-9]{2}!, \verb!/!, \verb![0-9]{4}!]] & $\rightarrow$ & \verb![0-9]{2}/([0-9]{2}/[0-9]{4})!
\end{tabular}\bigskip
\end{center}
%\captionsetup{labelformat=empty,belowskip=-15p%t}
%\caption{}
%\end{table}

\noindent
Analogously, to enumerate two capturing groups, consider two non-empty disjoint sub-lists. For the date regex, we can have at most 5 capturing groups because the atomic sub-regexes list has 5 elements. The following are some examples of multiple sub-lists of the atomic sub-regexes list and their resulting capturing groups:

%\begin{table}[h!]
%\centering
%\setlength{\extrarowheight}{10pt}
\begin{center}
\setlength{\extrarowheight}{10pt}
\begin{tabular}{rcl}
[[\verb![0-9]{2}!], \verb!/!, [\verb![0-9]{2}!], \verb!/!, \verb![0-9]{4}!] & $\rightarrow$ & \verb!([0-9]{2})/([0-9]{2})/[0-9]{4}! \\

[[\verb![0-9]{2}!], \verb!/!, [\verb![0-9]{2}!, \verb!/!, \verb![0-9]{4}!]] & $\rightarrow$ & \verb!([0-9]{2})/([0-9]{2}/[0-9]{4})! \\

[[\verb![0-9]{2}!], \verb!/!, [\verb![0-9]{2}!], \verb!/!, [\verb![0-9]{4}!]] & $\rightarrow$ & \verb!([0-9]{2})/([0-9]{2})/([0-9]{4})!
\end{tabular}\bigskip
\end{center}
%\captionsetup{labelformat=empty,belowskip=-15pt}
%\caption{}
%\end{table}
\end{example}

\section{Capturing Groups Synthesis}
The second component of the regex validation are capturing groups that reflect the captures provided with the valid examples. To synthesise these capturing groups, \Forest enumerates capturing groups over the produced regular expression as explained in \autoref{sec:cap_groups_enumeration}. Since the captured values are provided along with all valid examples, it is known how many capturing groups are required.

For each enumerated set of capturing groups, \Forest matches the regular expression with the capturing groups to each of the valid example strings. The enumeration process continues until a set of capturing groups that produces the user-provided captures is found. Once the correct captures are found, they are stored as part of the solution.

\begin{example}
Recall the example in \autoref{ex:1}: the user wished not only to validate the input string but also to extract some information from it. The goal was to extract the year from each date, so it could be used afterwards. In this situation, the examples provided by the user are:
%
\begin{multicols}{3}
    \begin{itemize}[label={},topsep=0em]
    \item 19/08/1996, 1996
    \item 26/10/1998, 1998
    \item 22/09/2000, 2000
    \item 01/12/2001, 2001
    \item 29/09/2003, 2003
    \item 31/08/2015, 2015
    \end{itemize}
\end{multicols}
%
As we saw back then, \Forest can provide capturing groups within the regular expression that allow the user to achieve this. In this example, we can see that each example is followed only by one capture, so we want to enumerate only one sub-list from the list of atomic regexes. In this specific case, the desired sub-list and corresponding capturing group are:
%
\begin{center}
\begin{tabular}{rcl}
[[\verb![0-9]{2}!], \verb!/!, \verb![0-9]{2}!, \verb!/!, \verb![0-9]{4}!] & $\rightarrow$ & \verb!([0-9]{2})/[0-9]{2}/[0-9]{4}!
\end{tabular}\bigskip
\end{center}
\end{example}


\section{Capture Conditions Synthesis}
The third component of the regex validation is a set of integer conditions over captured values in the example. The conditions are satisfied by all valid examples but not by any of the conditional invalid examples.

At this point, \Forest has a regular expression that matches all valid examples.
In order to compute the captures using the regular expression, we need all conditional invalid examples to be matched by this regular expression as well.
To ensure this, we remove from the set of conditional invalid examples all those that are not matched by the regular expression. 
This is equivalent to moving them from the conditional invalid set to the invalid set.
To the user, it matters only that all invalid and conditional invalid examples are rejected by the final regex validation; whether they are rejected because they do not match the regular expression or because they do not satisfy the capture conditions is not relevant. Therefore, this behaviour is inconsequential to the user.

To synthesise the capture conditions, \Forest starts by enumerating capturing groups using the enumeration process described in \autoref{sec:cap_groups_enumeration}. When synthesising capture conditions, the number of necessary capturing groups is not known beforehand. Thus, we enumerate capturing groups in progressively increasing number. We start by attempting to generate conditions with just one capturing group; if no solutions are found, enumerate two capturing groups, and so on.

Since it is our goal to synthesise integer conditions, we require that the captures resulting from any given set of capturing groups all correspond to integers. This is done by, for each enumerated capturing groups, matching the regex to all valid and conditional invalid examples and casting the resulting captures to integers. If this is not successful for all captures and all examples, these capturing groups cannot be used for capture conditions. The current capturing groups are discarded and the enumeration process continues.


For each enumerated set of integer capturing groups, \Forest tries to find a set of capture conditions over those capturing groups that are satisfied by all valid examples but not by any of the conditional invalid examples.
%
A capture condition is a 3-tuple. It contains a capture, an integer comparison operator and an integer argument. In this document, we use the notation \(\$i, i \in 0, 1, ...\) to reference the text captured by the \((i+1)\)\textsuperscript{th} group. In \Forest, we consider only two integer comparison operators, $\le$ and $\ge$. However, the algorithm can be expanded to include more integer comparison operators. We do not allow two capture conditions referring to the same capturing group that use the same operator, changing only the integer argument. Therefore, a capture condition can be identified by the capturing group's index and the integer comparison operator.

%\Forest uses \ac{SMT} to find which operators to use in the conditions for each capture, and which integer argument they should have.
We start by considering all possible capture conditions for a given set of capturing groups. Let \(\mathcal{C}\) be a set of capturing groups and \(\mathcal{C}(x)\) the captures that result from applying the capturing groups \(\mathcal{C}\) to example string \(x\). Let \(\mathcal{O}\) be the set of integer operators being considered; in \Forest \(\mathcal{O} = \{\le, \ge\}\). Then, the set of all capture conditions, \(\mathcal{D}\), is the combination of any capturing group with any operator: \(\mathcal{D} = \mathcal{C} \times \mathcal{O}\).


\Forest uses max-\ac{SMT} to select among all possible capture conditions the minimum set of capture conditions that are satisfied by all valid examples and by none of the conditional invalid.
%
Let \(\mathcal{V}\) be the set of all valid examples, \(\mathcal{I}\) the set of all conditional invalid examples and \(\mathcal{X} = \mathcal{V} \cup \mathcal{I}\) the set of all examples.
%\subsection{\ac{SMT} variables}
To encode the problem, we use the following sets of Boolean variables:
\begin{itemize}
    \item \(a_x\) for all examples \(x \in \mathcal{X}\).
    
    \(a_x = \) True means that example \(x\) satisfies all capture conditions in the solution.
    
    \item \(s_{\textit{cap}, x}\) for all captures \(\textit{cap} \in \mathcal{C}(x)\) and \(x \in \mathcal{X}\).
    
    \(s_{\textit{cap}, x} =\) True means that capture \textit{cap} in example \(x\) satisfies all used conditions that refer to it.

    
    \item \(u_{\textit{cond}}\) for all conditions \(\textit{cond} \in \mathcal{D}\). 
    
    \(u_{\textit{cond}} = \) True means condition \textit{cond} is used in the solution.
\end{itemize}

Besides these, we use a set of integer variables \(b_{\textit{cond}}\), for all conditions \(\textit{cond} \in \mathcal{D}\) that represent the integer argument present in each condition.


%\subsection{\ac{SMT} constraints}
Then, we need to define a set of constraints that ensure the set of selected captures correctly classifies the examples.
Let \(\textrm{SMT}(\textit{cond}, x)\) be the \ac{SMT} representation (in the underlying Linear Integer Arithmetic) of the condition \(\textit{cond}\) for example \(x\).
In this representation, the capture is an integer value, the integer operator is has the usual semantics and the integer argument is the corresponding \(b_{\textit{cond}}\) \ac{SMT} integer variable.

\begin{example}
Recall the first valid example from \autoref{ex:1} \(x_0\) = ``27/10/2017''.
%
Suppose \Forest has already synthesised the desired regular expression and enumerated the capturing group that corresponds to the day: \verb`([0-9]{2})/[0-9]{2}/[0-9]{4}`. 
%
Finally, consider the capture condition that refers to the first (and only) capturing group, \(\$0\) and operator \(\le\), represented by these two elements: \(c_0 = (\$0, \le\)).
%
The SMT representation for condition \(c_0\) and example \(x\) is then:

\[\textrm{SMT}(c_0, x_0)\; =\;  27 \le b_{c_0}\]
\end{example}

Constraint \ref{eq:cap_cond_s_def} states that a capture \textit{cap} in example \(x\) satisfies all conditions that refer to it if and only if for every condition that refers to that capturing group, either it is not used or it is satisfied for the value of that capture in that example.

\begin{equation}\label{eq:cap_cond_s_def}
    s_{cap,x} \leftrightarrow \bigwedge_{cond \in \mathcal{D}(cap)} (u_{cond} \rightarrow \textrm{SMT}(\textit{cond}, x))
\end{equation}

\begin{example}
Consider again \(x_0\) = ``27/10/2017'' and two capture conditions over \verb`([0-9]{2})/[0-9]{2}/[0-9]{4}`: \(c_0 = (\$0, \le\)) and \(c_1 = (\$0, \ge\)). Constraint \eqref{eq:cap_cond_s_def} for this example is then:
%
\begin{equation*}\label{eq:cap_cond_s_def}
    s_{0,x_0} \leftrightarrow (u_{c_0} \rightarrow 27 \le b_{c_0}) 
    \land (u_{c_1} \rightarrow 27 \ge b_{c_1})
\end{equation*}

\end{example}

Let \(\mathcal{C}(x)\) be the set of all captures resulting from applying the current capturing groups to example \(x\). Constraint \ref{eq:cap_cond_a_def} states that an example \(x\) satisfies all capture conditions that refer to it if and only if all the individual captures satisfy their respective conditions.

\begin{equation}\label{eq:cap_cond_a_def}
    a_x \leftrightarrow \bigwedge_{cap \in \mathcal{C}(x)} s_{cap,x}
\end{equation}

Finally, constraint \ref{eq:cap_cond_a_clause} ensures that all valid examples satisfy the used conditions and none of the conditional invalid examples does.

\begin{equation}\label{eq:cap_cond_a_clause}
    \bigwedge_{x \in \mathcal{V}} a_x \;\land\; \bigwedge_{x \in \mathcal{I}} \neg a_x
\end{equation}


Since we are looking for the minimum set of capture conditions that are satisfied by all valid examples and by none of the conditional invalid, we add soft clauses to penalise the addition of capture conditions to the solution:

\begin{equation}
    \bigwedge_{cond \in \mathcal{D}} \neg\, u_{cond}
\end{equation}

Running a \ac{SMT} solver with these constraints results in a solution. We consider part of the solution only the capture conditions whose \(u_cond\) is True in the resulting model. We also extract the values of the integer arguments in each condition from the model values of the \(b_{\textit{cond}}\) variables.

\section{Capture Conditions Disambiguation}

The same problem we encountered in synthesising a regular expression emerges for capture conditions: the ambiguity of the examples as a specification. There can be many sets of capture conditions that validate all valid examples and invalidate all condition invalid. To ensure the solution meets the user's needs, \Forest disambiguates the specification using a procedure based on distinguishing inputs similar to that used during the synthesis of the regular expression.

First, \Forest uses the \ac{SMT} solver to produce another valid solution. This is done, like before, by adding a clause that blocks the current model and then asking the solver to solve the new formula.
%
To block the current model, let \(\mathcal{D^+}\) be the set of conditions that were part of the solution in the current model, i.e., the set of conditions \(\textit{cond}\) for which \(u_{\textit{cond}}\) was assigned \true, and \(m(x)\) represent the value of variable \(x\) in the current model.
To get a different solution, we add constraint \ref{eq:cap_cond_soft} to ensure that in at least one of the selected captures, the integer value is different. Note that this allows for new models that use some of the same conditions with the same integer arguments for the conditions that they keep, but ensure that at least one condition is replaced by another.

\begin{equation}\label{eq:cap_cond_soft}
\bigvee_{\textit{cond} \in \mathcal{D^+}} b_{\textit{cond}} \ne m(b_{\textit{cond}})
\end{equation}

Another call to the \ac{SMT} solver supplies another, different solution. We have then \(\mathcal{S}_1\) and \(\mathcal{S}_2\), two different solutions, i.e., two different sets of capture conditions that satisfy the specification. Our goal is to find a distinguishing input: a string \(c\) which is satisfies all capture conditions in \(\mathcal{S}_1\), but not those in \(\mathcal{S}_2\), or vice-versa. First, to simplify the problem, \Forest eliminates from \(\mathcal{S}_1\) and \(\mathcal{S}_2\) conditions which are present in both: these are not relevant to compute a distinguishing input. Let \(\mathcal{S}_1^*\) (resp. \(\mathcal{S}_2^*\)) be the subset of \(\mathcal{S}_1\) (resp. \(\mathcal{S}_2\)) containing only the distinguishing conditions, i.e., the conditions that differ from those in  \(\mathcal{S}_2\) (resp. \(\mathcal{S}_1\)).

\begin{example}\label{ex:cond_cap_keep_distinct}
Recall the valid and conditional invalid examples from \autoref{ex:1}. Because there is no conditional invalid example that shows that there cannot be a date with the day 32, \Forest can produce two correct sets of capture conditions for the examples provided in  over \verb`([0-9]{2})/([0-9]{2})/[0-9]{4}`:
\begin{itemize}[nosep]
    \item \(\mathcal{S}_1 = \$0 \le 31 \wedge \$0 \ge 1 \wedge \$1 \le 12 \wedge \$1 \ge 1\)
    \item \(\mathcal{S}_2 = \$0 \le 32 \wedge \$0 \ge 1 \wedge \$1 \le 12 \wedge \$1 \ge 1\)
\end{itemize}
To find the correct solution, \Forest will first keep only the captures that differ from one solution to the other:
\begin{itemize}[nosep]
    \item \(\mathcal{S}_1^* = \$0 \le 31\)
    \item \(\mathcal{S}_2^* = \$0 \le 32\)
\end{itemize}
\end{example}

%To find a distinguishing input, we use the \ac{SMT} solver again.
Recall that \(\mathcal{S}_1\) and \(\mathcal{S}_2\) are capture conditions with respect to the same set of capturing groups \(\mathcal{C}\). The capture conditions do not refer to the whole matched string, but only to parts of it (the captured portions). Thus, we do not compute the distinguishing string \(c\) directly. Instead, we compute the integer value of the distinguishing captures: the captured parts of the distinguishing string that satisfy the only one of \(\mathcal{S}_1^*\) and \(\mathcal{S}_2^*\).
%
We define \(|\mathcal{C}|\) integer variables, \(c_i\), which correspond to the values of the distinguishing captures, i.e., the captures that result from applying the regular expression and its capturing groups to the distinguishing input string:
%
\[c_0, c_1, ..., c_{|\mathcal{C}|} = \mathcal{C}(c)\]

\begin{example}
In \autoref{ex:cond_cap_keep_distinct} we had two solutions: \(\mathcal{S}_1^* = \$0 \le 31\) and \(\mathcal{S}_2^* = \$0 \le 32\). Even though the regular expression had 2 capturing groups, since only one is present in the distinguishing conditions, we can define only one integer variable, \(c_0\), representing the integer value of the first capturing group in the distinguishing string.
\end{example}

We then define \(\textrm{SMT}(\textit{cond}, c)\) for the distinguishing string using the \(c_i\). In the \ac{SMT} representation of each condition \(\textit{cond}\), each capture represented by its respective \(c_i\) variable, the operator maintains it usual semantics and the integer argument is its value in the solution to which the condition belongs.

Finally, we add constraint \eqref{eq:cap_cond_distinguishing_constraint}, which states that \(c\) must satisfy one solution but not the other.

\begin{equation}\label{eq:cap_cond_distinguishing_constraint}
    \bigwedge_{\textit{cond} \in \mathcal{S}_1^*} \textrm{SMT}(\textit{cond}, c) \qquad\ne \bigwedge_{\textit{cond} \in \mathcal{S}_2^*} \textrm{SMT}(\textit{cond}, c)
\end{equation}

\begin{example}
Recall once more the two solutions from \autoref{ex:cond_cap_keep_distinct}, \(\mathcal{S}_1^* = \$0 \le 31\) and \(\mathcal{S}_2^* = \$0 \le 32\). To find \(c_0\), the value of the distinguishing capture for these solutions, we solve the constraint
\begin{equation*}
    \exists c_0 : c_0 \le 31 \ne c_0 \le 32
\end{equation*}
and get the value \(c_0 = 32\) for which the left-hand side of the constraint is False, and the right-hand side is True.
\end{example}

Note that string \(c\) is not a string variable: it is not part of the encoding and it is invisible to the solver. We include it here as an auxiliary structure. Therefore, the model that results from a call to the \ac{SMT} solver does not give us a distinguishing string \(c\). Instead, it outputs only values for the distinguishing captures \(c_i\). To produce the distinguishing string c, \Forest picks an example from the valid set, applies to it the regular expression with the capturing groups, and replaces its captures with the model values for \(c_i\). An additional step is required here to pad with zeros \(c_i\)'s integer values so that their string representation has the same length as the captures they are replacing in the valid string.






