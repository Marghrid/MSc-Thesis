\chapter{Background}\label{sec:background}
This section offers a brief introduction on some key notions required to understand the rest of this document.
We start off by looking into formal languages and associated definitions in \autoref{sec:languages}, followed by the closely related topic of regular expressions in \autoref{sec:regex}\footnote{Definitions and examples in Sections \ref{sec:languages} and \ref{sec:regex}
were, in part, adapted from Chapter~3 of \textit{Compilers: Principles, Techniques, and Tools}, by \citet*{DragonBook}.}. Finally, in \autoref{sec:ps}, we present an overview of the basic concepts of program synthesis.

\section{Formal Languages}\label{sec:languages}
Formal languages differ from the common meaning of the word \textit{language} in that they are built from a set of well-defined rules and thus stringently specified. Programming languages, such as \texttt{C} or \texttt{Python}, are examples of formal languages. They have a strict syntax and no deviation is tolerated: if a string does not comply with the syntax rules, it is not in the language.

To define a formal language, we must first define an alphabet.
An alphabet is a nonempty finite set whose elements are called the symbols, letters or tokens.
Symbols of the alphabet can be, for example, letters, digits, or punctuation marks. Alphabets are usually identified by capital Greek letters.

\begin{definition}[Alphabet]
An alphabet \(\Sigma\) is a nonempty finite set of symbols.
\end{definition}

\begin{example}
The following are examples of alphabets:

\begin{itemize}[topsep=0pt]
\item \(\Sigma_1 = \{0,1\}\): the binary alphabet,
\item \(\Sigma_2 = \{a,b,c,d,e,f,g,h,i,j,k,l,m,n,o,p,q,r,s,t,u,v,w,x,y,z\}\): the Latin lowercase alphabet.
\end{itemize}
\end{example}

Typically, the alphabets used in programming applications are larger than \(\Sigma_1\) or \(\Sigma_2\). The ASCII alphabet, for instance, contains 128 symbols which include all non-accentuated upper- and lower-case letters, digits, and most punctuation marks.

Given an alphabet, we can define strings (also called sentences or words) over it, which result from the concatenation of a finite number of the alphabet's symbols. Strings are usually identified by lower case letters, and written between single quotes. The length of a string \(s\), denoted \(|s|\), is the number of symbols in it.

\begin{definition}[String]
A string \(s\) over an alphabet \(\Sigma\) is a finite sequence of symbols drawn from \(\Sigma\).
\end{definition}

\begin{definition}[Length of a string]
The length of a string \(s\), \(|s|\), is the number of symbols in \(s\).
\end{definition}

\begin{example}
The following are examples of strings:
\begin{itemize}[topsep=0pt]
\item \(\epsilon\), the empty string, is the string of length zero,\par
\item \(s_1 = \) \textit{synthesis} is a string over alphabet \(\Sigma_2\) and it has length \(|s_1| = 9\).
\end{itemize}
\end{example}

At last, a language is a (finite or infinite) countable set of strings over some alphabet. Languages are usually identified by calligraphic capital letters.

\begin{definition}[Language]
A language \(\mathcal{L}\) is a countable set of strings over some fixed alphabet \(\Sigma\).
\end{definition}

\begin{example}
The following are examples of languages:
\begin{itemize}[topsep=0pt]
\item \(\emptyset\), the empty set, is a language which contains no strings,\par
\item \{\(\epsilon\)\} is the language that contains only the empty string.
\end{itemize}
\end{example}

\subsection{Regular operations}
Regular operations are a type of operation over languages, and regular languages are a subset of languages that are closed under regular operations. The most important regular operations are union, concatenation and closure.

The union of two languages \(\mathcal{L}\) and \(\mathcal{M}\), denoted \(\mathcal{L} \cup \mathcal{M}\), is the set of all strings in \(\mathcal{L}\) and all strings in \(\mathcal{M}\) (including those that are in both).
The concatenation of two languages \(\mathcal{L}\) and \(\mathcal{M}\), denoted \(\mathcal{L}\mathcal{M}\), is the set all strings formed by taking any string from \(\mathcal{L}\) and any string from \(\mathcal{M}\) and concatenating them. The concatenation of the same language, \(\mathcal{L}\), \(n\) times is sometimes denoted as \(\mathcal{L}^n\). For example  \(\mathcal{L}^3 = \mathcal{L}\mathcal{L}\mathcal{L}\).
The Kleene closure\footnotemark{} (also called Kleene star or simply star) of a language \(\mathcal{L}\), denoted \(\mathcal{L}^*\), is the set of strings resulting from the concatenation of any string in \(\mathcal{L}\) zero or more times. The Kleene closure of any language will always include the empty string.

\footnotetext{Kleene closure owes its name to the American mathematician Stephen Cole Kleene, who first described the concepts of regular language and regular expression in the~1950s.}
\begin{definition}[Regular operations] There are 3 regular operations on languages, union, concatenation and closure, which are defined as follows:
\begin{itemize}[topsep=.2em, itemsep=.2em]
\item Union: \(\mathcal{L} \cup \mathcal{M} = \{s :s \in \mathcal{L}\) \text{ or } \(s \in \mathcal{M}\}\),

\item Concatenation: \(\mathcal{L}\mathcal{M} =\{st : s \in \mathcal{L} \text{ and } t \in \mathcal{M}\}\),

\item Kleene closure: \(\mathcal{L}^*\) = \(\bigcup_{i=0}^\infty \mathcal{L}^i\).
\end{itemize}
\end{definition}

\section{Regular Expressions} \label{sec:regex}
Regular expressions are used to describe regular languages. The language defined by a regular expression \(r\) is represented as \(\mathcal{L}(r)\).
Regular expressions are built recursively out of smaller regular expressions connected by operators. The simplest regular expressions refer to languages that contain just one symbol and are represented as that symbol in monospaced font.

\begin{example}
The regular expression \regex{a} defines the language \(\mathcal{L}({\regexm{a}}) =\) \{\textit{a}\}.
\end{example}{}

All regular operations have a counterpart operation on regular expressions. For example, the union of two regular expressions \(r\) and \(q\) results in a regular expression that defines the regular language \(\mathcal{L}(r) \cup \mathcal{L}(q)\). 
Since regular languages are closed under regular operations, it is possible to apply these operations to any regular expression.
The operations in regular expressions are generally represented in the same way as their regular-language counterparts, with the exception of union, which is represented using a \(|\) instead of the usual set notation~\(\cup\). The operations on regular expressions and their equivalent regular operations  are defined in \autoref{tab:regex-ops}.

\begin{table}[t]
\centering

\newcolumntype{R}{>{\centering\arraybackslash}m{.25\textwidth}}
\newcolumntype{L}{>{\centering\arraybackslash}m{.24\textwidth}}
\newcolumntype{M}{>{\centering\arraybackslash}m{.33\textwidth}}

\begin{tabular}{@{}R L M@{}}
\toprule
Operation name & Operation on regular expressions & \quad Corresponding regular operation on languages \\ \midrule
Union          & \(r|q\) & \quad \(\mathcal{L}(r) \cup \mathcal{L}(q)\) \\
Concatenation  & \(rq\)  & \quad \(\mathcal{L}(r)\mathcal{L}(q)\) \\
Kleene closure & \(r^*\) & \quad \(\mathcal{L}(r)^*\)\\
%  &  Positive closure of \(\mathcal{L}\)                  & \quad \(\mathcal{L}^*\) = \(\cup_{i=1}^\infty \mathcal{L}^i\)                                                           \\ 
\bottomrule
\end{tabular}
\caption{Regular operations}
\label{tab:regex-ops}
\end{table}


The unary operator * has highest precedence, concatenation has second highest precedence and operator \(|\) has lowest precedence.
All three operators are left associative. Parentheses can be used to ensure a subexpression takes precedence over the rest.

\begin{example}
\regex{a|b*c} is a regular expression that defines the language
\[\mathcal{L}(\regexm{a|b*c}) = \{a\} \cup \{b\}^* \{c\}\]
over the alphabet \(\Sigma = \{a, b, c\}\). This language contains strings that are either a single \(a\) or zero or more \(b\)s followed by one \(c\).
\end{example}

If two regular expressions \(r\) and \(s\) denote the same regular language, we say they are equivalent and write \(r = s\). For instance, \regex{a|b} = \regex{b|a}. There are several algebraic laws that hold for arbitrary regular expressions, some of which are defined in \autoref{tab:regex-laws}.

\begin{table}[t]
\centering
\begin{tabular}{@{}c c@{}}
\toprule
Algebraic Law                      & \quad Description                                    \\ \midrule
\(r|s = s|r\)                      & \quad Union is commutative                           \\
\(r|(s|t) = (r|s)|t\)              & \quad Union is associative                           \\
\(r(st) = (rs)t\)                  & \quad Concatenation is associative                   \\
\(r(s|t) = rs|rt; (s|t)r = sr|tr\) & \quad Concatenations distributes over union          \\ 
%\(\epsilon r = r\epsilon = r\)     & \quad \(\epsilon\) is the identity for concatenation \\ 
%\(r^* = (r|\epsilon)^*\)           & \quad \(\epsilon\) is guaranteed in a Kleene closure \\ 
\(r|r = r\)      & \quad Union is idempotent \\
\(r^{**} = r^*\)   & \quad * is idempotent \\
\(r?? = r?\)       & \quad ? is idempotent \\
\(r^{++} = r^+\)   & \quad + is idempotent \\

\(r^{+*} = r^{*+} = r^*\)  & \quad .......... \\
\(r?^* = r^*? = r^*\)      & \quad .......... \\
\(r?^+ = r^+? = r^*\)      & \quad .......... \\ \bottomrule
\end{tabular}
\caption{Algebraic laws for regular expression operators}
\label{tab:regex-laws}
\end{table}

\medskip
Since the introduction of regular expressions with the basic operators for union, concatenation, and Kleene closure, some new operators have been defined for regular expressions with the purpose of easing the specification of certain string patterns. Here we show three of those commonly used operators: \(+\), \(?\), and character classes.

The unary postfix operator \(+\) represents positive closure, which can be interpreted as `one or more occurrences of' (note the similarity to Kleene closure, `zero or more instances of'). It can be defined as
\[r^+ = rr^* = r^*r,\]
and gives rise to a new algebraic law:
\[r^* = r^+|\epsilon.\]
%
The unary postfix operator \(?\) means `zero or one
occurrences of', and can be defined~as
\[ r? = r|\epsilon. \]
%
The operators \(+\) and \(?\) have the same precedence and associativity as operator~\(*\).


Character classes are a form of regular expression shorthand notation. A regular expression \(a_1|a_2|a_3|...|a_n\), where each \(a_i\) is a symbol of the alphabet, can be replaced by the shorthand \([a_1 a_2...a_n]\).
When \(a_1 a_2...a_n\) form a logical sequence, e.g., consecutive uppercase letters, lowercase letters, or digits, we
can replace them by \(a_1\mhyphen a_n\), that is, just the first and last tokens separated by
a hyphen.

%\mathchardef\mhyphen="2D

\begin{example}
The following are examples of character classes:
\begin{itemize}[]
\setlength\itemsep{1em}
\item \regex{[abc]} = \regex{a|b|c}
\item \regex{[a\mhyphen z]} = \regex{a|b|...|z}
\item \regex{[a\mhyphen z0\mhyphen 9]} = \regex{a|b|...|z|0|1|...|9}
\end{itemize}
\end{example}

\begin{comment}

\subsection{Regular Definitions}

For convenience, we may wish to give names to certain regular expressions and use those names in subsequent expressions, as if the names were themselves symbols. If \(\Sigma\) is an alphabet of basic symbols, then a regular definition is a sequence of definitions of the form:

\begin{equation}
\begin{gathered} % centered multiline equation.
d_1 \to r_1\\
d_2 \to r_2\\
...\\
d_n \to r_n
\end{gathered}
\end{equation}

\noindent
where each \(d_i\) is a new symbol, not in \(\Sigma\) and not the same as any of the other \(d\), and each \(r_i\) is a regular expression over the alphabet \(\Sigma \cup \{ d_1, d_2, ..., d_{i-1} \}\).

By restricting \(r_i\) to \(\Sigma\) and the previously defined \(d\)'s, we avoid recursive definitions, and we can construct a regular expression over \(\Sigma\) alone, for each \(r_i\).
We do so by first replacing uses of \(d_1\) in \(r_2\) (which cannot use any of the d's except
for \(d_1\)), then replacing uses of \(d_1\) and \(d_2\) in \(r_3\) by \(r_1\) and (the substituted) \(r_2\), and so on. Finally, in \(r_n\) we replace each \(d_i\), for \(i = 1, 2, ..., n-1\), by the substituted version of \(r_i\), each of which has only symbols of \(\Sigma\).

\begin{example}



We may wish to
\begin{equation*}
\begin{aligned}
\regex{number} &\to \regex{0|1|2|3|4|5|6|7|8|9} \\
%
\regex{lower\_case} &\to \regex{a|b|c|d|e|f|g|h|i|j|k|l|m|n|o|p|q|r|s|t|u|v|w|x|y|z} \\
%
\regex{upper\_case} &\to \regex{A|B|C|D|E|F|G|H|I|J|K|L|M|N|O|P|Q|R|S|T|U|V|W|X|Y|Z}\\
%
\regex{letter} &\to \regex{upper\_case | lower\_case}
\end{aligned}
\end{equation*}
\end{example}

\end{comment}

\begin{comment}


\subsection{Regex vs \acp{CFG}} \label{sec:regex-vs-CFG}

Grammars are a more powerful notation than regular expressions. Every construct that can be described by a regular expression can be described by a grammar, but not vice-versa. Alternatively, every regular language is a context-free language, but not vice-versa.

\begin{example}
Consider the regular expression \((a|b)^*abb\) and the \ac{CFG}:

\begin{equation}
\begin{aligned} % centered multiline equation.
A_0 &\to aA_0 | bA_0 | aA_1\\
A_1 &\to bA_2\\
A_2 &\to bA_3\\
A_3 &\to \epsilon
\end{aligned}
\end{equation}

The regular expression and the grammar describe the same language: the set of strings in \(\{a, b\}\) that end in \(abb\).

\end{example}

\begin{example}
The language \(\mathcal{L} = \{a^n b^n : n \ge 1\}\) is the language of strings in \(\{a, b\}\) that start with some \(a\)s followed by the same number of \(b\)s.  \(\mathcal{L}\) can be defined using a \ac{CFG} but not using a regular expression:  \(\mathcal{L}\) is a context-free language but not a regular language.
\end{example}

\end{comment}


\section{Program Synthesis} \label{sec:ps}
\begin{figure}[t]
  \centering
  \begin{tikzpicture}[%
  auto,
  trim left=(synth), trim right=(synth)
]

  \node[square] (synth) [yshift=-.2cm] {Synthesizer};
  
  \node[nosquare] (spec) [left=of synth, align=center] {Specification};
  
  \node[nosquare] (desired) [right=of synth] {\textbf{Desired Program}};

  \draw [arrow] (spec) to (synth);
  \draw [arrow] (synth) to (desired);
  
  \end{tikzpicture}

  \caption{Program Synthesis}
  \label{fig:program-synthesis}
\end{figure}
Program synthesis is the task of automatically generating a program that satisfies some desired behaviour expressed as a high-level specification~(see \autoref{fig:program-synthesis}).
This specification can range from a complete formal definition, such as a first-order formula \cite{DBLP:conf/ijcai/Green69,DBLP:conf/pldi/GulwaniJTV11}, to more ambiguous descriptions of the desired program's behaviour, like a set of input-output examples~\cite{balog2017deepcoder,DBLP:conf/pldi/FengMBD18,DBLP:conf/icse/JhaGST10,DBLP:conf/ijcai/ShawWG75,DBLP:journals/jacm/Summers77} or a natural language sentence~\cite{DBLP:conf/icse/DesaiGHJKMRR16,DBLP:journals/pacmpl/Yaghmazadeh0DD17}.

According to \citeauthor{PSnow}~\cite{DBLP:conf/ppdp/Gulwani10,PSnow}, a program synthesizer is typically characterised by three key dimensions:
\begin{enumerate*}[label=(\roman*)]
  \item the way the user specifies the desired characteristics of the program,
  \item the space of all possible programs the synthesizer can generate, and
  \item the search technique used to explore that space.
\end{enumerate*}

\paragraph{Desired behaviour specification} is the most important characteristic of a synthesizer from the users' point of view. It is the language in which they describe the behaviour of the program they intend to generate. It must take into consideration the underlying task, the users' technical background and which material is available when using the synthesizer.

\paragraph{Program space} is the set of all programs the synthesizer can possibly generate. In other words, it is the space over which the synthesizer will search for a feasible program. It depends on the domain of the problem the synthesizer is intended to solve. It must be expressive enough to ensure the desired program is included, but also restricted enough lest the search problem become intractable.

\paragraph{Search technique} refers to the method employed to find the desired program within the program space. It can be based on enumerative search, deductive search, constraint solving, or some combination thereof.

% \textcolor{purple}{It can be based on exhaustive search, version space algebras, machine learning techniques (such as belief propagation or genetic programming), or logical reasoning techniques.
% Most logical reasoning techniques involve two main steps: constraint generation, and constraint solving. (a) Constraint generation can be invariant-based, path-based, or input-based. (b) Constraint solving of resultant second-order quantified formulas typically involves reducing second-order unknowns to first-order unknowns (by use of templates), and eliminating universal quantifiers (by use of techniques such as Farkas lemma, cover algorithms, sampling), and then solving the resultant first-order quantifier-free constraints using off-the-shelf SAT/SMT solvers.}

\medskip

The three dimensions are further described in sections~\ref{sec:desired-behaviour-spec}, \ref{sec:program-space} and \ref{sec:search-technique}, respectively.

\subsection{Desired Behaviour Specification} \label{sec:desired-behaviour-spec}
For the synthesis procedure to start, the user must first specify the program's intended behaviour. The desired behaviour can be described in many different ways, and will internally be converted to some sort of behavioural constraints, which the output program must satisfy.

\begin{definition}[Desired Behaviour Specification]
The desired behaviour specification is a predicate \(\phi\), such that \(\phi(\vec{x}, y)\) is \true{} if and only if \(y\) is the desired output value for the input vector \(\vec{x}\).
\end{definition}


The first approaches to automating the creation of programs, proposed in \citeyear{DBLP:conf/ijcai/Green69} \cite{DBLP:conf/ijcai/Green69,DBLP:conf/ijcai/WaldingerL69}, were based in deductive synthesis.
Such synthesizers work based on a complete formal specification of the desired program's behaviour.
Then, they employ a theorem prover to construct a proof of the provided specification from which it is able to extract the executable program~\cite{DBLP:conf/ijcai/Green69,DBLP:journals/cacm/MannaW71,DBLP:journals/toplas/MannaW80,DBLP:conf/ijcai/WaldingerL69}.

Deductive synthesizers require the user to provide a complete formal description of the desired behaviour, such as a first-order formula.
Because it is a complete definition, the synthesizer always returns a program with the exact desired behaviour, and it is always satisfactory to the user.
However, writing these kind of specifications can be as complex as writing the program itself, and might force the user to learn a new formalism in any case.
\begin{figure}[t]
  \centering
  \begin{tikzpicture}[%
  auto
 % trim left=(synth), trim right=(synth)
]

  \node[square] (synth) [yshift=-.2cm] {Synthesizer};
  
  \node[enclosing_square, minimum height=1cm, align=left, rounded corners] (spec) [left=of synth] {\textcolor{gray!80}{Specification:}\\[.4em]\(\{ (\Vec{x_i}, y_i), i \in [1, N] \}\) };
  
  \node[nosquare] (desired) [right=of synth] {\textbf{Desired Program}};

  \draw [arrow] (spec) to (synth);
  \draw [arrow] (synth) to (desired);
  \end{tikzpicture}
  \caption{\ac{PBE}: the desired behaviour specification is a set of \(N\) input-output examples}
  \label{fig:pbe}
\end{figure}

Such difficulties motivated a new approach to program synthesis: inductive synthesis \cite{DBLP:conf/pldi/FengMBD18,DBLP:conf/pldi/FengMGDC17,DBLP:conf/ijcai/ShawWG75,DBLP:journals/jacm/Summers77}. The program is then built based on simpler (albeit ambiguous) specifications, easier for the user to devise. \acf{PBE} is a branch of inductive synthesis where the user intent is specified using input-output examples~\cite{DBLP:conf/pldi/GodefroidT12,DBLP:conf/ijcai/ShawWG75,DBLP:journals/jacm/Summers77,DBLP:conf/sigmod/WangCB17,DBLP:conf/pldi/WangCB17}, as shown in \autoref{fig:pbe}.
%\textcolor{red}{(\cite{DBLP:conf/ijcai/ShawWG75} synthesises simple LISP programs from one input-output example.~\cite{DBLP:journals/jacm/Summers77} from several output examples)} or natural language descriptions~\cite{DBLP:conf/icse/DesaiGHJKMRR16}).

\begin{definition}[Input-Output Examples]
Input-output examples are a type of incomplete specification defined as a set of \(N\) tuples:  \(\mathcal{X} = \{(\vec{x_i}, y_i), i \in \{1, ..., N\}\}\) where each \(y_i\) is the desired return value for input \(\vec{x_i}\).
\end{definition}

\begin{definition}[\acl{PBE}]
Programming by example is the problem of synthesising a program using input-output examples as a specification of the desired program behaviour. In programming by example, the behavioural constraint is
\[\bigwedge_{(x_i, y_i) \in \mathcal{X}} P(x_i) = y_i,\]
which states that program \(P\) is correct if it yields the correct output for the inputs specified in the specification.
%\[\phi(\vec{x}, y) = \neg (\exists z : (\vec{x}, z) \in \{(\vec{x_i}, y_i), i \in \{1, ..., N\}\} \wedge z \neq y)\]
\end{definition}

\begin{comment}
\begin{example}
Suppose we would like to generate a function which, for any integer input numbers \(x_1\) and \(x_2\) returns their sum:
\[f(x_1, x_2) = x_1 + x_2\]
A \ac{PBE} synthesiser would require some input-output examples as input, for example:
\[\{((1, 1), 2), ((1, 0), 1), ((2, 1), 3), ((2, 0), 2), ((1, 2), 3)\}\]
\end{example}
\end{comment}


Input-output examples, although easier for the users to obtain and comprehend, are an incomplete specification. In general, there are several programs consistent with the provided specification even though not all of those correspond to the desired program and might not exhibit the behaviour expected by the user for cases that are not covered by the specification.

The ambiguity of input-output examples raises the necessity of selecting one among multiple candidate programs. One way to do this selection is by ranking the correct programs according to the measure of some characteristic that is desirable in programs of the domain in question, and returning to the user the program that ranks the highest.
Common desirable characteristics used to rank the programs include: execution speed (when we are looking for an efficient program), robustness (how well it generalises to new input-output examples), and readability (uses common operations in the underlying language making it easy for the user to understand).
Another approach is to enable the synthesizer to interact with the user to try and disambiguate the underlying intent. Section~\ref{sec:user-interaction} provides more details on this topic.

\subsection{Program Space} \label{sec:program-space}
The next step towards finding a program that satisfies the user's needs is defining the space of programs over which the synthesizer will be performing the search: the program space.
We cannot consider all the programs that can be written using a full-featured programming language, as too many programs would be taken into consideration, rendering the search space intractable.
On that account, we need to restrict the language in which the programs are written in order to enable an efficient search of the program space.
However, we must ensure it remains expressive enough to capture many real-world tasks within the considered specialised domain.
We call this language \acf{DSL}.
The choice of a synthesizer's \ac{DSL} is crucial: it must allow a good balance between expressiveness and efficiency.

\begin{definition}[\acl{DSL}]
Domain specific language is the restricted language in which the synthesised programs are written. It includes information about both form (syntax) and meaning (semantics).
\end{definition}


One may impose restrictions on the allowed datatypes, as well as the operations over them so as to include only those relevant for the considered domain. For example, one could allow only integers and comparison operations, or arithmetic operations, or even only operations supported in some API exported by a given library. We can also restrict the program space by imposing constraints over the control structure of the program: we may disallow looping structures in the program, or bound the number of statements.

The constraints imposed on the language are named structural constraints, and they define the search space the synthesiser must consider. A \ac{CFG} is typically used to define the syntax of the \ac{DSL}.

\begin{definition}[Context-Free Grammar]
A \acl{CFG}~\cite{Jurafsky} is defined by a 4-tuple \((N, \Sigma, R, S)\), where:

\(N\) is a set of non-terminal symbols,

\(\Sigma\) is a set of terminal symbols (disjoint from \(N\)),

\(R\) is a set of rules or productions, each of the form \(A \to \beta\), where:

\quad \(A\) is a non-terminal,

\quad \(\beta\) is a string of symbols from the infinite set of strings \((\Sigma \cup N)*\), and

\(S\) is a designated start symbol and a member of \(N\).
\end{definition}


\acf{SyGuS} \cite{DBLP:conf/fmcad/AlurBJMRSSSTU13} is the branch of program synthesis in which the user to supplies the language syntax (in the form of a grammar) alongside the behaviour specification. This provides structure to the program space, which may allow for a more efficient search method. Furthermore, the generated programs are more interpretable to the user and better adapted to the domain at hand, since they are derived from the given grammar. On the other hand, \ac{SyGuS} requires the user to have a deeper technical knowledge not only on the specific domain on which he or she is working but also on formal languages and how to define them.

%Restricted DSLs can also enable more efficient special-purpose search algorithms. For example, if we consider a DSL that allows only the concatenation of substrings of an input string, a dynamic programming can be used to efficiently enumerate all possible outputs and thus search on such a space~\cite{DBLP:conf/oopsla/PolozovG15}.

% string programming/expression language that supports restricted forms of regular expressions, conditionals and loops. Expressive SQL queries

\subsection{Search Technique}\label{sec:search-technique}

Program synthesis can be seen as a search problem. It aims at finding a program in the search space defined by structural constraints that satisfies the behavioural constraints.

\begin{definition}[Program Synthesis]
Given a specification of desired behaviour \(\phi\), program synthesis is the problem of finding a program \(P\) that satisfies~\(\phi\).
\end{definition}


Several search techniques have been explored to solve program synthesis. In the remainder of this section, some of these search techniques are briefly explained. Note that one synthesizer does not have to apply only one of these techniques; more often, a combination of several techniques is applied.

\begin{figure}[t]
  \centering
  \pgfdeclarelayer{background}   % layer with the synthesizer rectangle
  \pgfsetlayers{background,main} % background behind main
  \begin{tikzpicture}[%
  auto,
  trim left=(verifier), trim right=(verifier)
] %, trim right=(spec)]

  \node (centre) at (0,0) {};

  \node[square] (enumerator) [above=of centre, yshift=-.2cm] {Enumerator};

  \node[square] (verifier) [below=of centre, yshift=.2cm] {Verifier};

  \node[nosquare] (DSL) [above=0.5 of enumerator] {\ac{DSL}};

  \node[nosquare] (spec)    [left=of verifier, xshift=-.8cm] {Specification};

  \node[nosquare] (desired) [right=of verifier, xshift=.8cm] {\textbf{Desired Program}};

  \draw [arrow] (enumerator.240) to node[left=.5mm,align=right,text width=1.5cm] (candidate) {\small Candidate program} (verifier.120);

  \draw [arrow, dashed] (verifier.60) to node[right=.5mm,align=left,text width=1.5cm]  (invalid) {\small Incorrect} (enumerator.300);

  \draw [arrow] (DSL) to (enumerator);

  \draw [arrow] (spec) to (verifier);
  \draw [arrow] (verifier) to node {Correct} (desired);
  
  \begin{pgfonlayer}{background}
    \path let \p1=(DSL.north west),
              \p2=(candidate.west)
    in
      coordinate (topleftcorner) at (\x2, \y1);
      
    \path let \p1=(verifier.south),
              \p2=(invalid.east)
    in
      coordinate (bottomrightcorner) at (\x2, \y1);
      
    \path let \p1=(topleftcorner),
              \p2=(bottomrightcorner)
    in
      coordinate (toprightcorner) at (\x2, \y1);
      
    \draw[enclosing_square] (synth)
      ($(topleftcorner)+(-0.6,0.6)$) rectangle
      ($(bottomrightcorner)+(0.6,-1)$);
    
    \node[gray,nosquare, anchor=north east] (synthtext) at ($(toprightcorner)+(0.5, 0.5)$) {Synthesizer};
  \end{pgfonlayer}

  \end{tikzpicture}

  \caption{Enumerative search}
  \label{fig:enumerative-search}
\end{figure}
\paragraph{Enumerative Search}\label{sec:enum-search} \cite{DBLP:conf/pldi/FengMGDC17,DBLP:conf/pldi/FengMBD18,DBLP:conf/gpce/LeeSO16,DBLP:conf/cp/OrvalhoTVMM19,DBLP:conf/cav/ReynoldsBNBT19} is a very common approach to solve the search problem of program synthesis. In this technique, we have two key components: an \textit{enumerator} and a \textit{verifier}.
The \textit{enumerator} successively enumerates programs from the program space in some order. For each candidate program, the \textit{verifier} subsequently checks whether it satisfies all behavioural constraints, i.e., it is consistent with the user intent specification.
If the \textit{verifier} redeems the program consistent to the given specification, then the program is correct and it can be returned to the user.
If, on the other hand, the \textit{verifier} decides that the given program does not satisfy the specification, then the \textit{enumerator} must pick a new program from the search space.
This loop is described in \autoref{fig:enumerative-search}.

A naive implementation of enumerative search does not scale to complex programs; however, it is a very effective strategy when coupled with some optimisation techniques. First off, the program space must be structured according to some metrics, usually program size or complexity, such that the \textit{enumerator} yields simpler programs first and only when those are refuted by the \textit{verifier} are more complex programs considered. This concept contributes to solving the problem of user intent ambiguity introduced in \autoref{sec:desired-behaviour-spec} by applying the Ockham's Razor principle: choose the simplest program that is consistent with the specification.

Another way to speed up enumerative search is by employing pruning techniques, such as discarding equivalent programs from the search space.
%The \textit{verifier} may also try to formulate a hint as to why a program deemed incorrect is not consistent with the specification and notify the \textit{enumerator}
In addition, when the \textit{verifier} rejects a program, it may apprise the \textit{enumerator} of the reason behind the program's incorrectness, thus reducing the program space and preventing the \textit{enumerator} from generating programs that fail in the same way.
This idea is further discussed in \autoref{sec:cegis}.

\paragraph{Deductive Search} \cite{DBLP:conf/oopsla/PolozovG15} is based on a top-down propagation of constraints through the grammar that defines the program space. It recursively reduces the problem of synthesising an expression that satisfies a certain specification to simpler sub-problems, thereupon combining those results.

Suppose we want to synthesise an expression \(e\) of the form \(F(e_1, e_2)\) that complies with a specification \(\phi\). Deductive search would leverage the inverse semantics of \(F\) to push constraints on \(e\) down through the grammar into constraints on \(e_1\) and \(e_2\). Finding \(e_1\) and \(e_2\) are then simpler sub-problems whose solutions, once found, can be combined to produce the originally desired expression \(e\).

Deductive search is very convenient when the underlying grammar allows for a rich set of constants. In such cases, enumerative search is no longer viable. A new enumeration step may be required for each possible constant in order to find the right one, resulting in too many iterations. On the other hand, the top-down deductive technique can deduce constants based on the accumulated constraints as the last  step in the search process.

\paragraph{Constraint Solving} \cite{Solar-LezamaPhDThesis,DBLP:conf/popl/SrivastavaGF10} consists on somehow generating a logical formula whose solution will yield the intended program, and then solving it.

Several approaches can be used to generate the formula. On one extreme, we have invariant-based methods, which will generate one formula that is satisfiable if and only if the program is consistent with the specification. Upon solving such a formula, we have not only a correct program but also a inductive proof of its correctness.
However, these methods do not scale well with the complexity of the specification --- the resulting formula may be intractable, and much more complex than the program itself.

To circumvent this drawback, we can instead make use of input-based methods. Then, the formula asserts the correctness of the program only on a subset of all possible inputs, which leads to much simpler constraints.

Once we have a logic formula, it must be solved. These formulas are of second-order logic, and need to be first reduced to first-order quantifier-free constraints that can the be solved using an off-the-shelf logic-based solver. Second-order reduction can be achieved using techniques such as \ac{CEGIS} \cite{Solar-LezamaPhDThesis}, which is described in \autoref{sec:cegis}.

