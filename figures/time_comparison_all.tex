\begin{figure}[t]
  \centering
  \begin{tikzpicture}
  \begin{axis}[
    xmode=linear,  ymode=linear,
    width=.6\linewidth, height=.5\linewidth,
    grid=major,
    %scale only axis,
    ymax=3600 ,xmax=64, xmin=1, ymin=0.05,
    xtick distance=8, ytick distance=600,
    ylabel = {\footnotesize{Time (s)}},
    xlabel = {\footnotesize{}{Instances solved}},
    tick label style = {font=\scriptsize},
    % legend columns=3, 
    % legend style={
    %   anchor=north,
    %   at={(.5,-.2)},
    %   %draw=none,
    %   /tikz/column 2/.style={
    %     column sep=2ex
    %   },
    %   /tikz/column 4/.style={
    %     column sep=2ex
    %   }
    % },
    legend style={
      anchor=west,
      at={(1.05,.5)}
    },
    cycle list/Dark2-6,
    cycle multi list={Dark2-6}  % plot colors
  ]
  \addplot+[mark=asterisk, fill opacity=1, draw opacity = 1] table [x=ranking, y=time, col sep=comma] {data/ranking_lines.csv};
  \addlegendentry{\footnotesize Line-based};

  \addplot+[mark=asterisk, fill opacity=1, draw opacity = 1] table [x=ranking, y=time, col sep=comma] {data/ranking_ktree.csv};
  \addlegendentry{\footnotesize \(k\)-tree};

  \addplot+[mark=star, fill opacity=1, draw opacity = 1, mark size=1.5] table [x=ranking, y=time, col sep=comma] {data/ranking_dynamic.csv};
  \addlegendentry{\footnotesize Dyn. Multi-tree};

  \addplot+[mark=x, fill opacity=1, draw opacity = 1] table [x=ranking, y=time, col sep=comma] {data/ranking_resnax.csv};
  \addlegendentry{\footnotesize \Regel};
  
  \addplot+[mark=+, fill opacity=1, draw opacity = 1] table [x=ranking, y=time, col sep=comma] {data/ranking_nopruning.csv};
  \addlegendentry{\footnotesize No pruning};
  
  \addplot+[mark=Mercedes star, fill opacity=1, draw opacity = 1] table [x=ranking, y=time, col sep=comma] {data/ranking_multitree.csv};
  \addlegendentry{\footnotesize Multi-tree};
\end{axis}
\end{tikzpicture}
%\caption{Average time necessary to generate a question using the $2$ interaction models with the $3$ implementations.}
\caption{Instances solved using different methods}
\label{fig:comparison_all_methods}
\end{figure}